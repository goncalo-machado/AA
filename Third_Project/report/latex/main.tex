\documentclass[]{revdetua}
\usepackage{graphicx} % Required for inserting images

\Header{Volume}{7}{janeiro}{2024}{0}

\title{Advanced Algorithms - Third Project \linebreak Most Frequent Letters}
\author{Gonçalo Machado Nmec 98359}
\date{January 2024}

\begin{document}

\maketitle

\begin{abstract}
This report studies different methods to identify the most frequent letters in text files. This problem is addressed by the course ”Advanced Algorithms” at the University of Aveiro. The proposed approaches are Exact Counters, Approximate Counters, and the Misra-Gries algorithm, also known as Frequent-Count algorithm. Along with their implementation, we evaluate the quality of estimates
regarding the exact counts and analyze the computational efficiency and limitations of the developed approaches. The chosen coding language was Python.
\end{abstract}

\begin{resumo}% Note: in Portuguese
Este relatório apresenta diferentes métodos de identificar as letras mais frequentes em ficheiros de texto. Este problema é abordado pelo curso "Algoritmos Avançados" na Universidade de Aveiro.Os métodos desenvolvidos são Contadores Exatos, Contadores Aproximados e o algoritmo Misra-Gries, também conhecido como algoritmo Frequent-Count .A linguagem de programação escolhida foi Python.
\end{resumo}

\begin{keywords}
Most Frequent Letters, Text Processing, Exact Counter, Approximate Counter, Misra-Gries Algorithm, Frequent-Count
\end{keywords}

\begin{palavraschave}
Letras Mais Frequentes, Processamento de Texto, Contador Exato, Contador Aproximado, Algoritmo Misra-Gries, Frequent-Count
\end{palavraschave}

\section{Introduction}
This report is part of the third project of "Advanced Algorithms". The goal of the project was to identify the most frequent letters in text files (literary works) using different methods, and to evaluate the quality of estimates regarding the exact counts.

The following sections describe what each of the three methods used ( Exact Counters, Approximate Counters, and the Misra-Gries algorithm, also known as Frequent-Count algorithm) are and how they work. 

In each section pertaining a different approach, an analysis of the computation efficiency and limitations of the developed approach will be done in terms of absolute and relative errors, average values, among other relevant parameters. 

A comparison with other relevant approaches will also be done to see if the same most frequent letters were identified and in the same relative order.We will also see if the most frequent letters are similar in text files of the same literary work, but in different languages.

Alongside the report there were also the files containing code for each of the algorithms developed, the code used for comparing the algorithms, the text files used and the results that were obtained from running the algorithms and performing the comparison between them. In order to run the code that generates the counters, the following commands should be used:

\$ python exact\_counter.py 

\$ python approximate\_counter.py

\$ python frequent\_counter.py

For generating the files containing the comparison between counter, the following command should be used:

\$ python compare\_counters.py 

\section{Text Processing}

For this project, as explained before, we were asked to identify the most frequent letters in text files of literary works, and to see if the same letters were the most frequent in different languages. For this reason, we picked three text files containing the literary work "The Odyssey" by Homer in English, Spanish and French. The files were obtained from the Project Gutenberg website. Project Gutenberg is a volunteer effort to digitize and archive cultural works, as well as to encourage the creation and distribution of eBooks.

Before the files could be used, they needed to be processed. The first step was to remove extra text in the text files so that only the literary work would be present. This meant removing the \textbf{Project Gutenberg} file headers (which are added to the beginning of each ebook to provide additional information about the book), but also removing text that was present in some languages but not others (such as prefaces or footnotes). 

Although the file headers are easily spotted due to some keywords, the text that existed in some languages but not all were not as easy to remove. To remove them, we used some of the words in the first sentence of the literary work to mark the beginning and some words from the last sentence to mark the end, and removed everything that was not between the markers.

The next step of the processing was to remove all stop words and punctuation, as well as some keywords that were an indication of illustrations, thus not belonging to the literary work text. Stop words are common words that do not contribute much meaning to the text, such as articles, conjunctions, and prepositions. The stop words of each language were obtained from the \textbf{CountWordsFree} website. \textbf{CountWordsFree} is a web-based service for content writers, web developers, and professionals who provide search ranking optimization services.

The final step was to remove all numbers and non-alphabetic characters, leaving only the letters of the alphabet, and converting every letter to uppercase, to ensure that the counters do not count a lower case letter as being different from the same letter in upper case.

\begin{table}[!ht]
    \centering
    Text Length Before and After Processing
    \begin{tabular}{|l|l|l|l|}
    \hline
        Language & Initial Length & Final Length & Difference \\ \hline
        English & 698069 & 216616 & 481453 \\ \hline
        Spanish & 1001728 & 367748 & 633980 \\ \hline
        French & 668432 & 330184 & 338248 \\ \hline
    \end{tabular}
\end{table}

As we can see, after the processing of the literary works, the length was cut in more than half, with the length of the Spanish language text file ending up with a third of the original length. 

\section{Exact Counter}

In the world of text analysis, exact counters, also known as exact frequency counters, play a vital role in quantifying the occurrences of letters within a text file. These tools are essential for revealing linguistic patterns and extracting valuable insights. Exact frequency counters are valuable for understanding the distribution of letters in a given text. They find applications in linguistic studies, language modeling, and cryptographic tasks.

Exact counters excel in accurately counting individual letter occurrences, allowing for nuanced linguistic analysis and providing insights into the structural aspects of language through the examination of letter frequencies. Researchers leverage the versatility of exact frequency counters not just for linguistic exploration but also as effective tools for deciphering encrypted messages using frequency analysis techniques.

However, it’s worth noting that exact counters can be slower and require more memory to store the count of each item. It is a limitation that becomes more significant if the dataset is large or if the processing speed is a concern.

After executing the code that contains the exact counter algorithm for each of the three text files, we were able to identify the most frequent letters in each language:
\begin{enumerate}
    \item English: 'E': 28671, 'S': 19490, 'A': 16673, 'R': 15432, 'N': 14411, 'O': 13811
    \item Spanish: 'A': 44356, 'E': 42152, 'O': 31916, 'R': 28115, 'S': 27850, 'I': 23203
    \item French:  'E': 41711, 'S': 28511, 'R': 27794, 'A': 25607, 'I': 25110, 'N': 22881
\end{enumerate}

\section{Approximate Counter}

Approximate counters are specialized data structures designed to provide an estimated count of item frequencies within a dataset. These counters prioritize efficiency over absolute accuracy, enabling them to process large datasets more swiftly than exact counters.

In the context of identifying letters in a text file, we can utilize approximate counters to gauge the frequency of each letter. This method serves practical purposes similar to exact counters, such as analyzing letter distributions and identifying common patterns in written communication. To implement an approximate counter for this task, we create a counter for each letter and increment the corresponding counter with a given probability when encountering a letter, which for this project was 1/4, or 25\%. This meant that the counter would increment once every 4 encounters, approximately. 

One advantage of using approximate counters is their faster processing speed and lower memory requirements compared to exact counters. However, it's important to note that approximate counters trade off some accuracy, providing an estimate of item frequencies rather than an exact count. This may be less suitable for applications requiring precise counts.

The following results were obtaining after using the approximate counters in 20000 trials.

\begin{table}[!ht]
    \centering
    Absolute and Relative Errors For English Version\linebreak\linebreak
    \begin{tabular}{|l|l|}
    \hline
        ~ & English \\ \hline
        Expected value & 54154.0 \\ \hline
        Variance & 27077.0 \\ \hline
        Standard deviation & 164.55090397806995 \\ \hline
        Mean absolute error & 161.37945 \\ \hline
        Mean relative error & 0.2980009786904006\% \\ \hline
        Mean accuracy ratio & 100.00171797097168\% \\ \hline
        Smallest counter value & 53244 \\ \hline
        Largest counter value & 54921 \\ \hline
        Mean counter value & 54154.93035 \\ \hline
        Mean absolute deviation & 161.38131070000023 \\ \hline
        Standard deviation & 202.20370990384353 \\ \hline
        Maximum deviation & 910.9303500000024 \\ \hline
        Variance & 40886.34029887771 \\ \hline
    \end{tabular}
\end{table}

\begin{table}[!ht]
    \centering
    Absolute and Relative Errors For Spanish Version\linebreak\linebreak
    \begin{tabular}{|l|l|}
    \hline
        ~ & Spanish \\ \hline
        Expected value & 91937.0 \\ \hline
        Variance & 45968.5 \\ \hline
        Standard deviation & 214.4026585656064 \\ \hline
        Mean absolute error & 208.94865 \\ \hline
        Mean relative error & 0.22727373092444111\% \\ \hline
        Mean accuracy ratio & 100.0020301402047\% \\ \hline
        Smallest counter value & 90879 \\ \hline
        Largest counter value & 92936 \\ \hline
        Mean counter value & 91938.86645 \\ \hline
        Mean absolute deviation & 208.93907735999977 \\ \hline
        Standard deviation & 262.5453762959799 \\ \hline
        Maximum deviation & 1059.8664500000014 \\ \hline
        Variance & 68930.07461439769 \\ \hline
    \end{tabular}
\end{table}

\begin{table}[!ht]
    \centering
    Absolute and Relative Errors For French Version\linebreak\linebreak
    \begin{tabular}{|l|l|}
    \hline
        ~ & French \\ \hline
        Expected value & 82546.0 \\ \hline
        Variance & 41273.0 \\ \hline
        Standard deviation & 203.15757431117353 \\ \hline
        Mean absolute error & 197.9248 \\ \hline
        Mean relative error & 0.23977515567077662\% \\ \hline
        Mean accuracy ratio & 100.00142647735808\% \\ \hline
        Smallest counter value & 81476 \\ \hline
        Largest counter value & 83517 \\ \hline
        Mean counter value & 82547.1775 \\ \hline
        Mean absolute deviation & 197.91635449999842 \\ \hline
        Standard deviation & 247.7787363632118 \\ \hline
        Maximum deviation & 1071.1775000000052 \\ \hline
        Variance & 61394.30219375002 \\ \hline
    \end{tabular}
\end{table}

\begin{table}[!ht]
    \centering
     Absolute and Relative Errors For Letter E (English)\linebreak\linebreak
    \begin{tabular}{|l|l|}
    \hline
        ~ & English \\ \hline
        Expected value & 7167.75 \\ \hline
        Variance & 3583.875 \\ \hline
        Standard deviation & 59.86547419005382 \\ \hline
        Mean absolute error & 58.276225 \\ \hline
        Mean relative error & 0.8130337274598115\% \\ \hline
        Mean accuracy ratio & 99.98529873391232\% \\ \hline
        Smallest counter value & 6894 \\ \hline
        Largest counter value & 7470 \\ \hline
        Mean counter value & 7166.69625 \\ \hline
        Mean absolute deviation & 58.27169162499992 \\ \hline
        Standard deviation & 73.11573897552825 \\ \hline
        Maximum deviation & 303.30375000000004 \\ \hline
        Variance & 5345.91128593758 \\ \hline
    \end{tabular}
\end{table}

\begin{table}[!ht]
    \centering
    Absolute and Relative Errors For Letter A (Spanish)\linebreak\linebreak
    \begin{tabular}{|l|l|}
    \hline
        ~ & English \\ \hline
        Expected value & 11089.0 \\ \hline
        Variance & 5544.5 \\ \hline
        Standard deviation & 74.46139939592862 \\ \hline
        Mean absolute error & 72.92875 \\ \hline
        Mean relative error & 0.657667508341596\% \\ \hline
        Mean accuracy ratio & 100.00761971322933\% \\ \hline
        Smallest counter value & 10703 \\ \hline
        Largest counter value & 11423 \\ \hline
        Mean counter value & 11089.84495 \\ \hline
        Mean absolute deviation & 72.9274825749998 \\ \hline
        Standard deviation & 91.34607823818999 \\ \hline
        Maximum deviation & 386.8449500000006 \\ \hline
        Variance & 8344.106009497526 \\ \hline
    \end{tabular}
\end{table}

\begin{table}[!ht]
    \centering
    Absolute and Relative Errors For Letter E (French)\linebreak\linebreak
    \begin{tabular}{|l|l|}
    \hline
        ~ & English \\ \hline
        Expected value & 10427.75 \\ \hline
        Variance & 5213.875 \\ \hline
        Standard deviation & 72.207167234285 \\ \hline
        Mean absolute error & 70.12445 \\ \hline
        Mean relative error & 0.6724792021289306\% \\ \hline
        Mean accuracy ratio & 99.99845268634174\% \\ \hline
        Smallest counter value & 10101 \\ \hline
        Largest counter value & 10826 \\ \hline
        Mean counter value & 10427.58865 \\ \hline
        Mean absolute deviation & 70.12432091999963 \\ \hline
        Standard deviation & 88.04118150716432 \\ \hline
        Maximum deviation & 398.4113500000003 \\ \hline
        Variance & 7751.249641177455 \\ \hline
    \end{tabular}
\end{table}

\begin{table}[!ht]
    \centering
    Accuracyfor Different Languages
    \begin{tabular}{|l|l|l|l|}
    \hline
        Language (Accuracy) & English & Spanish & French \\ \hline
        Letter Order & 20.18\% & 0\% & 0.42\% \\ \hline
        Top 3 Letter Order & 100.0\% & 100.0\% & 96.05\% \\ \hline
        Top 1 Letter & 100.0\% & 100.0\% & 100.0\% \\ \hline
    \end{tabular}
\end{table}

The tables show statistical values for the different language versions (English, German, French). These values include measures of central tendency (mean, expected value), dispersion (variance, standard deviation, mean absolute deviation), and accuracy (mean absolute error, mean relative error, mean accuracy ratio).

In general, the results are that despite the randomness inherent of the approximate counters, the accuracy is extremely high in all languages, as proven by the mean relative error and mean absolute error, with the mean relative error being less than 1\% in all languages.

As for identifying the order of the most frequent letters, we see that the algorithm has an 100\% accuracy identifying the most frequent letter in all languages, and 100\% accuracy identifying the three most frequent letter in order in nearly all languages, French being the exception. This exception is due to the fact that the second and third most frequent letters are really close in terms of counters, which makes it more vulnerable to the randomness of the algorithm.

The results were not as promising when identifying the order of the most frequent letters, with the English language having a 20\% accuracy and the other two languages 0\% accuracy. Although it is clearly possible to see that this algorithm is not good for identifying the order of most frequent letters, an explanation for the Spanish and French languages accuracy exists.

These languages have letters with accents and can include letters of other alphabets other than the Latin alphabet. This letters were a very small number of times when compared with the Latin alphabet letters and are very close between them in terms of counters, which makes them even more vulnerable to the randomness of the algorithm.

\section{Misra-Gries Algorithm (Frequent-Count)}

The Misra-Gries algorithm, also recognized as Frequent-Count, stands as a renowned approach for determining item frequencies in a data stream. Operating as a one-pass algorithm, it efficiently processes the data stream in a single run. Simultaneously, it remains a straightforward yet effective algorithm applicable across diverse fields, such as natural language processing and database indexing.

A variant of the Misra-Gries algorithm is the version with a parameter k, targeting the identification of the (k-1) most frequent items in a data stream. This involves managing (k - 1) counters, each corresponding to the k-1 most frequent items at any given time. Incrementing the relevant counter upon processing an item is the core mechanism. The parameter k governs result accuracy, maintaining, at most, (k - 1) candidates simultaneously and ensuring the inclusion of items with a frequency of at least m / k.

By extending its capability to detect the (k - 1) most frequent items, the Misra-Gries algorithm with k offers enhanced insights into the element distribution in the stream.

The Misra-Gries algorithm excels in efficiency, processing large data streams within milliseconds, making it well-suited for real-time applications. Its implementation is straightforward, requiring a minimal number of counters and basic arithmetic operations.

Nonetheless, the Misra-Gries algorithm has its limitations. Being an approximate algorithm, it furnishes frequency estimates rather than precise counts. While this trade-off is crucial for achieving heightened efficiency, it may not be optimal for applications requiring exact counts.

For this project, we use this algorithm with k = 3, k = 5 and k = 10.

\begin{table}[!ht]
    \centering
    Accuracy for the Misra-Gries algorithm
    \begin{tabular}{|l|l|l|l|}
    \hline
        Language & k & Correct Letters & Accuracy \\ \hline
        English & 3 & 1/2 & 50.00\% \\ \hline
        English & 5 & 1/4 & 25.00\% \\ \hline
        English & 10 & 4/9 & 44.44\% \\ \hline
        Spanish & 3 & 0/2 & 0.00\% \\ \hline
        Spanish & 5 & 2/4 & 50.00\% \\ \hline
        Spanish & 10 & 6/9 & 66.67\% \\ \hline
        French & 3 & 0/2 & 0.00\% \\ \hline
        French & 5 & 1/4 & 25.00\% \\ \hline
        French & 10 & 6/9 & 66.67\% \\ \hline
    \end{tabular}
\end{table}

As it is possible to see, the accuracy of the algorithm increases as k increases, although it is still low overall. For k = 3, only in the English language it had a non 0\% accuracy. For k = 4, every language had at least one correct letter. For k = 10, we see a bigger accuracy, with most of the languages getting more than half the correct letters. This shows, as expected, that the accuracy of the algorithm is directly tied to the k value. 


\section{Processing Time}

\begin{table}[!ht]
    \centering
    Exact Counters Time
    \begin{tabular}{|l|l|}
    \hline
        Language & Time \\ \hline
        English & 0.004995107650756836 \\ \hline
        Spanish & 0.008999109268188477 \\ \hline
        French & 0.010001182556152344 \\ \hline
    \end{tabular}
\end{table}

\begin{table}[!ht]
    \centering
    Approximate Counters (Average) Time
    \begin{tabular}{|l|l|}
     \hline
        Language & Time \\ \hline
        English & 0.03644352322816849 \\ \hline
        Spanish & 0.06214751232862473 \\ \hline
        French & 0.05643431528806686 \\ \hline
    \end{tabular}
\end{table}

\begin{table}[!ht]
    \centering
    The Frequent Count Algorithm Time
    \begin{tabular}{|l|l|l|}
    \hline
        Language & k & Time \\ \hline
        English & 3 & 0.09251284599304199 \\ \hline
        English & 5 & 0.09551620483398438 \\ \hline
        English & 10 & 0.08251547813415527 \\ \hline
        Spanish & 3 & 0.15652132034301758 \\ \hline
        Spanish & 5 & 0.1650235652923584 \\ \hline
        Spanish & 10 & 0.136519193649292 \\ \hline
        French & 3 & 0.1420297622680664 \\ \hline
        French & 5 & 0.1465141773223877 \\ \hline
        French & 10 & 0.12402200698852539 \\ \hline
    \end{tabular}
\end{table}

By analysing the tables, we can see that the Frequent Count algorithm was the slowest of the three, with the approximate counter coming in second and the fastest overall being the exact counters. This is likely due to the extra operations that are done in the approximate counters (the randomness aspect) and the Misra-Gries algorithm (decreasing all the counters and deleting if count reaches 0).

\section{Conclusion}

In general, the choice between using an exact counter, an approximate counter, or the Misra-Gries algorithm depends on the specific requirements of the task at hand. If precise counts are necessary, exact counters are the better choice. However, if memory efficiency is more important, approximate counters or the Misra-Gries algorithm may be the better choice.

The Exact Counter was the best overall, in terms of speed and accuracy. Its simplicity and low operations allowed it to be superior than the other algorithms.

The Approximate Counter were very capable of getting the three most frequent letters in their relative order, and the results were very accurate, but it was slower than the exact counter and had very low accuracy when obtaining the order of the most frequent letters.

The Misra-Gries algorithm was generally the worst, both in terms of time and in terms of accuracy, although if a higher k is used the results would be better and it might make the algorithm worth choosing over other options.

\begin{thebibliography}{00}

\bibitem{b1} Wikipedia (2024). \textit{Approximate Counting Algorithm} [Online]. Available: \url{https://en.wikipedia.org/wiki/Approximate_counting_algorithm}
\bibitem{b2} Wikipedia (2024) \textit{Misra–Gries summary} [Online]. Available: \url{https://en.wikipedia.org/wiki/Misra%E2%80%93Gries_summary}
\bibitem{b3} Wikipedia (2024) \textit{Stop words} [Online]. Available: \url{https://countwordsfree.com/stopwords}
\bibitem{b4} Gutenberg \textit{Project Gutenberg} [Online]. Available: 
\url{https://www.gutenberg.org/}

\end{thebibliography}

\end{document}